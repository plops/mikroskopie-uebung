\documentclass{article}
\usepackage[ngerman]{babel}
\usepackage{amsmath}
\usepackage{amsopn}
\usepackage{graphicx}
\usepackage{amssymb}
\usepackage{geometry}
\usepackage{pdfpages}
\usepackage{units}
%\usepackage[utf8]{inputenc}
%\usepackage[T1]{fontenc}
%\usepackage{textcomp}
%\usepackage{gensymb}
\geometry{verbose,a4paper,tmargin=-4mm,bmargin=0mm,lmargin=15mm,rmargin=12mm}

\DeclareMathOperator{\rect}{rect}
\begin{document}
\author{Martin Kielhorn, Rainer Heintzmann}
\title{\"Ubung 6 Licht-Mikroskopie (Prof. Heintzmann)}
\maketitle
\noindent Diese \"Ubung ist abzugeben am 19.~Jan.~2015.

\section{Transferfunktionen verschiedener Mikroskoptypen}
\subsection*{Fluoreszenzbildgebung in $I^5M$}
Skizzieren Sie die Transferfunktion eines Fluoreszensmikroskops vom I5M-Typ.

\subsection*{Theoretische Aufl\"osunggrenzen}

Was ist die maximale Aufl\"osung (im $\mathbf{k}$-Raum und im Ortsraum
als feinstes Objekt-Gitter) m\"oglich f\"ur Bildgebung nach der ersten
Born'schen N\"aherung und bei Fluoreszenzbildgebung mit r\"aumlich
konstanter Beleuchtung und r\"aumlich variierender Beleuchtung? Wie ist
es m\"oglich die letztere Grenze trotzdem zu umgehen?


\section{Strukturierte Beleuchtung}

Wir betrachten die Bildentstehung in einem Fluoreszensmikroskop mit
strukturierter Beleuchtung durch Zweistrahlinterferenz von
koh\"arentem Licht. Es werden vier Bilder mit verschiedenen
Phasenbeziehungen zwischen den beiden Beleuchtungsstrahlen
erstellt. Der Einfachheit halber sei der Phasenunterschied $\phi$
zwischen den Beleuchtungsstrahlen $0, 2\pi/4, 2\pi\cdot 2/4$ und
$2\pi\cdot 3/4$.


\begin{figure}[htbp]
  \centering
  \input{sim-mixing.pdf_tex}
  \caption{\"Uberlagerung der drei Beugungsordnungen die aus einem
    durch Zweistrahlinterferenz strukturiert beleuchtetem
    Fluoreszensobjekt austreten (nach
    \cite{heintzmann1999laterally}).}
  \label{fig:sim-mixing}
\end{figure}


\paragraph{a)} 
Skizzieren Sie, ausgehend von Abbildung \ref{fig:sim-mixing}, die
drei weiteren Ortsfrequenzdiagramme f\"ur die anderen Beleuchtungsphasen.

\paragraph{b)} Wie erh\"alt man aus den vier Bildern $I_{0\ldots 3}$
mit unterschiedlichen Beleuchtungsphasen die 0te Ordnung?

\paragraph{c)} Geben Sie eine M\"oglichkeit an, um aus den Bildern die
1te Ordnung zu bestimmen.

\paragraph{d)} Stellen Sie ein Bildmodell der folgenden Form auf:
\begin{align}
\begin{pmatrix} I_0 \\ I_1 \\ I_2 \\ I_3 \end{pmatrix} &=
M 
\begin{pmatrix} C_{-1} \\ C_0 \\ C_1 \end{pmatrix} 
\end{align}
Schreiben sie die Elemente der Matrix $M$ auf. Wie kann man im
allgemeinen nach den Komponenten $C_{-1}$, $C_0$ und $C_1$ aufl\"osen?
Wie ber\"ucksichtigt man in diesem Fall eine nichtquadratische Matrix
$M$?


\bibliographystyle{plain}
\bibliography{bib}


\end{document}
