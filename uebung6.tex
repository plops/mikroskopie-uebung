\documentclass{article}
\usepackage[ngerman]{babel}
\usepackage{amsmath}
\usepackage{amsopn}
\usepackage{graphicx}
\usepackage{amssymb}
\usepackage{geometry}
\usepackage{pdfpages}
\usepackage{units}
%\usepackage[utf8]{inputenc}
%\usepackage[T1]{fontenc}
%\usepackage{textcomp}
%\usepackage{gensymb}
\geometry{verbose,a4paper,tmargin=25mm,bmargin=20mm,lmargin=15mm,rmargin=20mm}

\DeclareMathOperator{\rect}{rect}
\begin{document}
\author{Martin Kielhorn, Rainer Heintzmann}
\title{\"Ubung 6 Licht-Mikroskopie (Prof. Heintzmann)}
\maketitle
\noindent Diese \"Ubung ist abzugeben am 19.~Jan.~2015.

\section{Transferfunktionen verschiedener Mikroskoptypen}
\subsection*{Fluoreszenzbildgebung in $I^5M$}
Skizzieren Sie die Transferfunktion eines Fluoreszensmikroskops vom I5M-Typ.

\subsection*{Theoretische Aufl\"osunggrenzen}

Was ist die maximale Aufl\"osung (im $\mathbf{k}$-Raum und im Ortsraum als feinstes Objekt-Gitter) m\"oglich f\"ur Bildgebung nach der ersten Born'schen Näherung und bei Fluoreszenzbildgebung mit räumlich konstanter Beleuchtung und räumlich variierender Beleuchtung? Wie ist es möglich die letztere Grenze trotzdem zu umgehen?


\section{Strukturierte Beleuchtung}

\begin{figure}[htbp]
  \centering
  \input{sim-mixing.pdf_tex}
  \caption{\"Uberlagerung der drei Beugungsordnungen in einem
    strukturiert beleuchtetem Fluoreszens-Mikroskopbilds (nach \cite{heintzmann1999laterally}).}
  \label{fig:sim-mixing}
\end{figure}

\bibliographystyle{plain}
\bibliography{bib}


\end{document}
