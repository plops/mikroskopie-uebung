\documentclass{article}
\usepackage{amsmath}
\usepackage{amsopn}
\usepackage{graphicx}
\usepackage{amssymb}
\usepackage{geometry}
\usepackage{pdfpages}
\usepackage{units}
%\usepackage[utf8]{inputenc}
%\usepackage[T1]{fontenc}
%\usepackage{textcomp}
%\usepackage{gensymb}
\geometry{verbose,a4paper,tmargin=25mm,bmargin=20mm,lmargin=15mm,rmargin=20mm}

\DeclareMathOperator{\rect}{rect}
\begin{document}
\author{Martin Kielhorn}
\title{\"Ubung 4 Licht-Mikroskopie (Prof. Heintzmann)}
\maketitle
\noindent Diese \"Ubung ist abzugeben am 8.~Dez.~2014.

\section{Abbe-Limit f\"ur Weitfeld-Fluoreszensmikroskop}
Ermitteln Sie die z-Aufl\"osung ($k_{z\,\textrm{max}}$) eines
Weitfeld-Fluoreszensmikroskops mit halben Aperturwinkel
$\alpha=67^\circ$, Immersionsindex $n=1.518$ und Wellenl\"ange
$\lambda=\unit[520]{nm}$.

\begin{figure}[htbp]
  \centering
  \input{missing-cone.pdf_tex}
  \caption{Geometrie zur Bestimmung des Supports der OTF f\"ur ein Weitfeld-Mikroskop.}
  \label{fig:missing-cone}
\end{figure}


\section{Abbe-Limit f\"ur konfokales Fluoreszensmikroskop}
\subsection*{a) infinitesemal kleines Pinhole}
Ausgehend von $h_\textrm{ges}=h_\textrm{ex}\cdot h_\textrm{em}$ und
damit
$\widetilde h_\textrm{ges}=\widetilde h_\textrm{ex}\otimes\widetilde
h_\textrm{em}$
bestimmen Sie $k_{x\,\textrm{max}}$ und $k_{z\,\textrm{max}}$ f\"ur
ein konfokales Fluoreszensmikroskop. Die Parameter seien: halber
Aperturwinkel $\alpha=67^\circ$, Immersionsindex $n=1.518$,
Anregungswellenl\"ange $\lambda_\textrm{ex}=\unit[488]{nm}$ und
Detektionswellenl\"ange $\lambda_\textrm{em}=\unit[520]{nm}$.
\subsection*{b) Pinholegr\"o\ss e eine Airy Einheit}
Was \"andert sich, wenn das Pinhole auf eine Airy Einheit
($\unit[1]{AU}=1.22\lambda/\textrm{NA}$) ge\"offnet wird?

\section{Transmissionsmikroskop Streuung, Bornsche N\"aherung}
Zeichnen Sie den Support der Objektraumfrequenzen f\"ur ein
Transmissionsmikroskop mit Beleuchtungsapertur
$\textrm{NA}_\textrm{ill} = 0.2$ und Detektionsapertur
$\textrm{NA}_\textrm{ill} = 0.5$.
\section{Reflexionsmikroskop Streuung, Bornsche N\"aherung}
Zeichnen Sie den Support der Objektraumfrequenzen f\"ur ein
Reflexionsmikroskop mit Beleuchtungsapertur
$\textrm{NA}_\textrm{ill} = 0.2$ und Detektionsapertur
$\textrm{NA}_\textrm{ill} = 0.5$.

\end{document}
