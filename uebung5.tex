\documentclass{article}
\usepackage{amsmath}
\usepackage{amsopn}
\usepackage{graphicx}
\usepackage{amssymb}
\usepackage{geometry}
\usepackage{pdfpages}
\usepackage{units}
%\usepackage[utf8]{inputenc}
%\usepackage[T1]{fontenc}
%\usepackage{textcomp}
%\usepackage{gensymb}
\geometry{verbose,a4paper,tmargin=25mm,bmargin=20mm,lmargin=15mm,rmargin=20mm}

\DeclareMathOperator{\rect}{rect}
\begin{document}
\author{Martin Kielhorn}
\title{\"Ubung 5 Licht-Mikroskopie (Prof. Heintzmann)}
\maketitle
\noindent Diese \"Ubung ist abzugeben am 15.~Dez.~2014.

\section{Transferfunktionen verschiedener Mikroskoptypen}
\subsection{Konfokales Fluoreszensmikroskop}
Skizzieren Sie die Transferfunktion eines konfokalen Fluoreszensmikroskops.
\subsection{I5M}
Skizzieren Sie die Transferfunktion eines Fluoreszensmikroskops vom I5M-Typ.
\subsection{4Pi}
Skizzieren Sie die Transferfunktion eines inkoherenten 4Pi-Mikroskops.

\end{document}
