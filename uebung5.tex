\documentclass{article}
\usepackage{amsmath}
\usepackage{amsopn}
\usepackage{graphicx}
\usepackage{amssymb}
\usepackage{geometry}
\usepackage{pdfpages}
\usepackage{units}
%\usepackage[utf8]{inputenc}
%\usepackage[T1]{fontenc}
%\usepackage{textcomp}
%\usepackage{gensymb}
\geometry{verbose,a4paper,tmargin=25mm,bmargin=20mm,lmargin=15mm,rmargin=20mm}

\DeclareMathOperator{\rect}{rect}
\begin{document}
\author{Martin Kielhorn, Rainer Heintzmann}
\title{\"Ubung 5 Licht-Mikroskopie (Prof. Heintzmann)}
\maketitle
\noindent Diese \"Ubung ist abzugeben am 22.~Dez.~2014.

\section{Transferfunktionen verschiedener Mikroskoptypen}
\subsection*{Konfokales Fluoreszenzmikroskop}
Skizzieren Sie die Transferfunktion eines konfokalen Fluoreszensmikroskops.
\subsection*{Fluoreszenzbildgebung in $I^5M$}
Skizzieren Sie die Transferfunktion eines Fluoreszensmikroskops vom I5M-Typ.
\subsection*{Fluoreszenzbildgebung in 4Pi}
Skizzieren Sie die Transferfunktion eines inkoherenten 4Pi-Mikroskops.

\subsection*{Theoretische Aufl\"osunggrenzen}

Was ist die maximale Aufl\"osung (im $\mathbf{k}$-Raum und im Ortsraum als feinstes Objekt-Gitter) m\"oglich f\"ur Bildgebung nach der ersten Born'schen Näherung und bei Fluoreszenzbildgebung mit räumlich konstanter Beleuchtung und räumlich variierender Beleuchtung? Wie ist es möglich die letztere Grenze trotzdem zu umgehen?


\end{document}
