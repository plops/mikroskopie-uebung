\documentclass{article}
\usepackage{amsmath}
\usepackage{amsopn}
\usepackage{graphicx}
\usepackage{amssymb}
\usepackage{geometry}
\usepackage{pdfpages}
\usepackage{units}
\geometry{verbose,a4paper,tmargin=25mm,bmargin=20mm,lmargin=15mm,rmargin=20mm}

\DeclareMathOperator{\rect}{rect}
\begin{document}
\author{Martin Kielhorn}
\title{\"Ubung 3 Licht-Mikroskopie (Prof. Heintzmann)}
\maketitle
\noindent Diese \"Ubung ist abzugeben am 24.~Nov.~2014.

\section{Strahlenoptik am Mikroskopobjektiv}
Bestimmen Sie den Durchmesser der hinteren Brennebene (back focal
plane) eines Zeiss-Mikroskopobjektivs mit Vergr\"o\ss erung $M=20$ und
$\textrm{NA}=0.55$ (Luft). Nutzen Sie dabei die Zeiss-Standard Tubusl\"ange
$f_\textrm{TL}=\unit[164.5]{mm}$ um auf die Brennweite des Objektivs
zu kommen.

\section{Herschelbedingung}
Alle von der optischen Achse ausgehenden Strahlen sollen unabh\"angig
von ihrem Winkel $\phi$ eine longitudinale Vergr\"osserung $M$
erfahren. Leiten Sie diese sogennante Herschelbedingung her. Kann man
diese Bedingung f\"ur beliebig gro\ss e Winkel $\phi$ erf\"ullen?

\section{Wellenoptik am Mikroskopobjektiv}
Lesen Sie den ersten Teil der angeh\"angten Arbeit von A.~Sch\"onle
und S.~Hell. Ermitteln Sie einen Ausdruck f\"ur die optische
Transferfunktion (OTF) der Detektion eines Fluoreszens-Mikroskops der
nur noch ein Linienintegral enth\"alt.

Vernachl\"assigen Sie alle vektoriellen Abh\"angigkeiten und gehen Sie
von einer homogenen Pupillenfunktion aus. Das heisst setzen Sie in
Formel $(1)$ der Arbeit $a_f(\vartheta,\varphi)=1$ und
$P(\vartheta,\varphi)=\Theta(\alpha-\vartheta)$. Wobei $\alpha$ den halben
\"Offnungswinkel des Objektiv beschreibt und $\Theta(x)$ sei die
Heaviside Stufenfunktion:
\begin{align}
  \Theta(x)&=\begin{cases}
  0, & x<0 \\
  1, & x \ge 0
  \end{cases}
\end{align}
Das Objektiv sei ein Luftobjektiv $(n=1)$.

\subsection*{Zusatz:}
Berechnen Sie numerisch einen transversalen Schnitt entlang
$\mathbf{k}=(k_x,0,0)$ durch die OTF und stellen Sie das Ergebnis graphisch dar. % $\lambda_{det}=\unit[500]{nm}$.

\includepdf[pages={1,2,3}]{2002schoenle.pdf}
\end{document}
